\documentclass[10pt, compress]{beamer}

	\usetheme[usetitleprogressbar]{m}


	\usepackage{booktabs}
	\usepackage{amsmath}
	\usepackage{nicefrac}
	\usepackage{color}
	\usepackage{wrapfig}
	\usepackage{tikz}


	\title{Un fablab au Mans}
	\subtitle{}
	\date{16 Novembre 2015}
	\author{HAUM --- Mathieu Gaborit}
	\institute{}

\begin{document}
	\maketitle

	\begin{frame}
		\frametitle{Qui Parle ?}

		\begin{block}{Mathieu Gaborit}
			\begin{itemize}
				\item Co-fondateur du \alert{HAUM} (Hackerspace Au Mans)
				\item Étudiant en Acoustique à l'\alert{Université du Maine}
				\item A travaillé pour la CCI dans le cadre du \alert{projet de fablab}
			\end{itemize}
		\end{block}

		\begin{block}{HAUM -- Hackerspace Au Mans}
				\begin{tikzpicture}[remember picture,overlay]
					\draw (current page.east) + (-3,0) node { \includegraphics[width=0.4\textwidth]{logo_haum}};
				\end{tikzpicture}
				\begin{itemize}
				\item \alert{30+} membres
				\item \alert{3 ans} d'existence
				\item \alert{50} dépôts de code open-source sur Github
				\item \alert{4} formations sur Arduino, Git, etc...
				\item \alert{Projets divers} : opendata, lumière, musique, hack, etc...
			\end{itemize}


		\end{block}
	\end{frame}

	\begin{frame}
		\frametitle{Au Menu}
	  \setbeamertemplate{section in toc}[sections numbered]
		\tableofcontents
	\end{frame}


	\section{Qu'est ce qu'un fablab ?}
	\begin{frame}
		\frametitle{Fablab : Définition}
		\begin{quote}
			Un fab lab (contraction de l'anglais fabrication laboratory, «\,laboratoire de fabrication\,») est un lieu ouvert au
			public où il est mis à sa disposition toutes sortes d'outils pour la conception et la réalisation d'objets.\\
			\begin{flushright}
			--- \textbf{Wikipedia}
			\end{flushright}
		\end{quote}
		\pause

		\begin{center}
			\large Fablab = \alert{lieu d'échanges et d'innovation}
		\end{center}
	\end{frame}

	\begin{frame}
		\frametitle{Fablab : le nécessaire}

		Petite recette pour un fablab semi-public :

		\pause

		\begin{itemize}[<+->]
			\item Un \alert{lieu}
			\item Des \alert{équipements}
			\item Un/Des animateur(s)
			\item Des opérations, des objets de \alert{communication}
			\item Un \alert{contexte} économique et éducatif \alert{favorable}
		\end{itemize}
	\end{frame}

	\begin{frame}
		\frametitle{Une orientation économique}

		Possibilité d'amener le lieu vers les entreprises :
		\begin{itemize}
			\item Mise à disposition d'\alert{outils de prototypage} à moindre coût (mutualisés)
			\item Organisation de \alert{workshops}, de conférences
			\item \alert{Mise en relation} d'entrepeneurs dans un cadre de travail
		\end{itemize}


		\pause

		\begin{tikzpicture}[remember picture, overlay]
			\draw (current page.north east) + (-1.5,-2.5) node {\includegraphics[width=0.1\textwidth]{french_tech.png}};
		\end{tikzpicture}
		\begin{center}
			\large Important pour l'obtention du label \alert{FrenchTech}
		\end{center}
	\end{frame}

	\section{Points Importants}

	\begin{frame}
		\frametitle{Atouts pour l'éducation}

		\begin{itemize}
			\item \alert{Outil de travail} pour les étudiants
			\item Atelier équipé nécessaire à la \alert{réalisation d'essais}
			\item Prise de conscience sur les \alert{techniques de fabrication}
			\item \alert{Rencontre} avec des entreprises
			\item \alert{Formation} sur le terrain
		\end{itemize}
	\end{frame}

	\begin{frame}
		\frametitle{Atouts pour les entreprises}

		\begin{itemize}
			\item \alert{Réduction des coûts} du prototypage \& du développement
			\item Accès ponctuel à des outils exceptionnels pour \alert{prototyper}
			\item Rencontre avec d'autres entreprises, d'\alert{autres compétences}
			\item Favorise les interactions entre recherche et entreprise dans un cadre plus informel
			\item \alert{Formation continue} des équipes
		\end{itemize}
	\end{frame}

	\section{Etat du projet au Mans}

	\begin{frame}
		\frametitle{Une autre approche : le fablab distribué}

		\begin{center}
			\alert{Idée :} Relier les projets existants et créer le lieu complémentaire des autres
		\end{center}

		\pause
		\begin{itemize}
			\item Réduction des coûts
			\item Inclusion dans un \alert{écosystème plus global}
			\item Mise en place de \alert{partenariats}
			\item \alert{Échanges d'idées}, de savoir faire
			\item Création d'une \alert{offre cohérente} sur Le Mans
		\end{itemize}
	\end{frame}

	\begin{frame}
		\frametitle{Les partenaires -- Les Lieux}

		\begin{itemize}
			\item \alert{Créalab} Petit usinage, conception, usinage propre
			\item \alert{Les Subsistances} Usinage grand volume, métallerie
			\item \alert{La Ruche Numérique} Accompagnement des start-uppers
			\item \alert{HAUM} Accompagnement des particuliers
		\end{itemize}

		\pause

		Il manque donc : un lieu pour les usinages de moyen format, équipé pour :

		\pause

		\begin{itemize}[<+->]
			\item \alert{workshops}
			\item \alert{prototypage} électronique (PCB, soudure, mesure)
			\item \alert{prototypage} mécanique simple (tournage/fraisage, découpe, etc..)
			\item \alert{conception}
		\end{itemize}
	\end{frame}

	\begin{frame}
		\frametitle{Timeline}

		\begin{tikzpicture}[overlay, remember picture]
			\draw (current page.north east) + (-2,-2.5) node {\includegraphics[width=0.25\textwidth]{logo_ruche.png}};
			\draw (current page.south east) + (-2.5,+2) node {\includegraphics[width=0.2\textwidth]{logo_crealab.png}};
		\end{tikzpicture}

		\begin{itemize}
			\item \textbf{Nov. 2012} Début du projet --- Création du \alert{HAUM}
			\item \textbf{Mars 2013} Achat d'une \textbf{imprimante 3D} par la Ruche Numérique
			\item \textbf{Mai~ 2014} Mise à disposition à la Ruche d'\alert{une salle} pour le HAUM
			\item \textbf{Juin 2014} Création d'une \alert{mission} au sein de la \alert{CCI} pour réfléchir à la création d'un fablab
			\item \textbf{Sep. 2015} Renouveau du projet --- contacts avec \alert{Créalab} \& des entrepeneurs
			\item \textbf{Nov. 2015} On y est.
		\end{itemize}
	\end{frame}

	\section{Et maintenant ?}
	\begin{frame}
		\frametitle{Todo List}
		\begin{itemize}
			\item Définir les termes des partenariats
			\item Discuter des conditions d'accès aux différents lieux
			\item Informer les réseaux d'entreprises
			\item Choisir une cible et les équipements associés
		\end{itemize}
	\end{frame}

	\plain{Merci !\\
	\vspace{0.25\textwidth}
	Des questions ?\\
	\vspace{0.15\textwidth}
\scriptsize{\texttt{mathieu@haum.org}}\hspace{1cm}\large{|}\hspace{1cm}\scriptsize{\texttt{contact@haum.org}}
	}


\end{document}

% vim: ts=2 sw=2
